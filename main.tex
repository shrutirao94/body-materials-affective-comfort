%% The first command in your LaTeX source must be the \documentclass command.
%%%% Small single column format, used for CIE, CSUR, DTRAP, JACM, JDIQ, JEA, JERIC, JETC, PACMCGIT, TAAS, TACCESS, TACO, TALG, TALLIP (formerly TALIP), TCPS, TDSCI, TEAC, TECS, TELO, THRI, TIIS, TIOT, TISSEC, TIST, TKDD, TMIS, TOCE, TOCHI, TOCL, TOCS, TOCT, TODAES, TODS, TOIS, TOIT, TOMACS, TOMM (formerly TOMCCAP), TOMPECS, TOMS, TOPC, TOPLAS, TOPS, TOS, TOSEM, TOSN, TQC, TRETS, TSAS, TSC, TSLP, TWEB.
% \documentclass[acmsmall]{acmart}

%%%% Large single column format, used for IMWUT, JOCCH, PACMPL, POMACS, TAP, PACMHCI
% \documentclass[acmlarge,screen]{acmart}

%%%% Large double column format, used for TOG
% \documentclass[acmtog, authorversion]{acmart}

%%%% Generic manuscript mode, required for submission
%%%% and peer review
\documentclass[acmconf, anonymous, review]{acmart}
\usepackage{todonotes}
\usepackage{xcolor}

%% Fonts used in the template cannot be substituted; margin 
%% adjustments are not allowed.
%%
%% \BibTeX command to typeset BibTeX logo in the docs
\AtBeginDocument{%
  \providecommand\BibTeX{{%
    \normalfont B\kern-0.5em{\scshape i\kern-0.25em b}\kern-0.8em\TeX}}}

%% Rights management information.  This information is sent to you
%% when you complete the rights form.  These commands have SAMPLE
%% values in them; it is your responsibility as an author to replace
%% the commands and values with those provided to you when you
%% complete the rights form.
\setcopyright{acmcopyright}
\copyrightyear{2018}
\acmYear{2018}
\acmDOI{XXXXXXX.XXXXXXX}

%% These commands are for a PROCEEDINGS abstract or paper.
\acmConference[Conference acronym 'XX]{Make sure to enter the correct conference title from your rights confirmation emai}{June 03--05,
  2018}{Woodstock, NY}


\acmBooktitle{Woodstock '18: ACM Symposium on Neural Gaze Detection, June 03--05, 2018, Woodstock, NY} 
\acmPrice{15.00}
\acmISBN{978-1-4503-XXXX-X/18/06}


%%
%% Submission ID.
%% Use this when submitting an article to a sponsored event. You'll
%% receive a unique submission ID from the organizers
%% of the event, and this ID should be used as the parameter to this command.
%%\acmSubmissionID{123-A56-BU3}

%%
%% end of the preamble, start of the body of the document source.
\begin{document}

%%
%% The "title" command has an optional parameter,
%% allowing the author to define a "short title" to be used in page headers.
\title{Affective Comfort in Built Hybrid Learning Spaces}

%%
%% The "author" command and its associated commands are used to define
%% the authors and their affiliations.
%% Of note is the shared affiliation of the first two authors, and the
%% "authornote" and "authornotemark" commands
%% used to denote shared contribution to the research.
\author{Shruti Rao}
\email{s.rao@uva.nl}
\orcid{1234-5678-9012}
\affiliation{
  \institution{University of Amsterdam}
  \city{Amsterdam}
  \country{The Netherlands}
}



%%
%% By default, the full list of authors will be used in the page
%% headers. Often, this list is too long, and will overlap
%% other information printed in the page headers. This command allows
%% the author to define a more concise list
%% of authors' names for this purpose.
\renewcommand{\shortauthors}{Rao et al.}

%%
%% The abstract is a short summary of the work to be presented in the
%% article.
\begin{abstract}
%   Abstracts should be about 150 words.
Given that people spend a significant amount of time within built spaces, designing spaces considering the impact that it may have on occupants’ comfort and emotions is an important challenge for Human Building Interaction (HBI). In this position paper, we suggest that materiality of tangible and intangible artifacts in a hybrid learning space will play a key role in comfort of occupants. To that end, we describe our case study and how artifacts identified through the study can serve as the first step towards designing comfort-enabling materials in a smart hybrid environment. 
\end{abstract}

%%
%% The code below is generated by the tool at http://dl.acm.org/ccs.cfm.
%% Please copy and paste the code instead of the example below.
%%

\begin{CCSXML}
<ccs2012>
 <concept>
  <concept_id>10010520.10010553.10010562</concept_id>
  <concept_desc>Computer systems organization~Embedded systems</concept_desc>
  <concept_significance>500</concept_significance>
 </concept>
 <concept>
  <concept_id>10010520.10010575.10010755</concept_id>
  <concept_desc>Computer systems organization~Redundancy</concept_desc>
  <concept_significance>300</concept_significance>
 </concept>
 <concept>
  <concept_id>10010520.10010553.10010554</concept_id>
  <concept_desc>Computer systems organization~Robotics</concept_desc>
  <concept_significance>100</concept_significance>
 </concept>
 <concept>
  <concept_id>10003033.10003083.10003095</concept_id>
  <concept_desc>Networks~Network reliability</concept_desc>
  <concept_significance>100</concept_significance>
 </concept>
</ccs2012>
\end{CCSXML}

\ccsdesc[500]{Computer systems organization~Embedded systems}
\ccsdesc[300]{Computer systems organization~Redundancy}
\ccsdesc{Computer systems organization~Robotics}
\ccsdesc[100]{Networks~Network reliability}

%%
%% Keywords. The author(s) should pick words that accurately describe
%% the work being presented. Separate the keywords with commas.
\keywords{affective computing, built environments, hybrid learning spaces, learning, materiality, body perception}

%% A "teaser" image appears between the author and affiliation
%% information and the body of the document, and typically spans the
%% page.
% \begin{teaserfigure}
%   \includegraphics[width=\textwidth]{sampleteaser}
%   \caption{Seattle Mariners at Spring Training, 2010.}
%   \Description{Enjoying the baseball game from the third-base
%   seats. Ichiro Suzuki preparing to bat.}
%   \label{fig:teaser}
% \end{teaserfigure}

 
%%
%% This command processes the author and affiliation and title
%% information and builds the first part of the formatted document.
\maketitle

\section{Introduction}
% smart buildings + hybrid 
``When the machine becomes a building" - the evolution of information and communication technologies (ICT) and building energy management systems (BEMS) have resulted in buildings transforming from inanimate structures into digital, interactive objects \cite{nembrini2017human}. These ``smart built environments" that encompass intelligent artifacts have presented an opportunity for Human Computer Interaction (HCI) research to find solutions for humans to sustainably co-exist within such buildings. This becomes especially important when smart and functionally hybrid buildings replace traditional learning spaces where learning, working, and leisure are all expected to co-exist \footnote{https://lab42.uva.nl/}. We colloquially refer to these non-traditional spaces as \textcolor{red}{hybrid learning spaces}. 

% comfort
A key aspect of human experience within built environments (and by extension smart environments) is derived from the physical and emotional comfort of the occupants \cite{alavi2017comfort}. This subjective comfort which is largely as a result of qualities of the built environment can play a role in impacting the occupants awareness and behaviour within that environment.  

% position
In this position paper, we consider the materiality of a hybrid learning space that encapsulates both tangible (for eg., hardware, digital systems, constructed elements, furniture, etc.) and intangible artifacts (for eg., air quality, sound, touch and temperature). We posit that the design considerations for hybrid learning spaces needs to be examined through the lens of material experience design. We believe that material properties of artifacts can be used to positively influence occupants comfort (affective and physical), and consequently their learning behaviours and attitudes in smart, hybrid learning spaces.

% We posit that materiality needs to be examined through the lens of smart, hybrid learning spaces, and designed in a manner that positively influences occupants subjective comfort (affective and physical), and consequently their learning behaviours and attitudes.

% paper outline
The remainder of this paper provides a background on human building interaction (HBI), an emerging field of research in tangent to HCI. We then describe our case study as a first step towards understanding the materiality of the artifacts in the hybrid learning space, and the contributions our findings can make towards the material experiences and design community. 


\section{Background}
% HBI
Human building interaction (HBI) is a burgeoning area of research that focuses on capturing, understanding and enhancing human interactions and experiences both with and within ``smart built environments" \cite{alavi2016future}. The primary goal of HBI being a framework that can be used to understand, compare, and converge research efforts from the fields of HCI, design, and Architecture in envisioning and shaping the future of living spaces and all that they encompass \cite{nembrini2017human, alavi2018artifacts}. 

% Comfort and Affect in Built Environment
The concept of comfort is central to occupants within built environments. Comfort is understood in the form of occupants' physiological and emotional (affective) responses to the built environment \cite{alavi2017comfort}. HBI especially examines the relationship between occupant comfort and four physical characteristics of the indoor environment - temperature, air, light and sound \cite{hawkes2007environmental, bluyssen2009indoor}. 

% Examples of existing work
Much of the work in comfort studies focus on designing ``optimally informative" systems such as temperature calendars, air quality forecasts, noise level indicators, and wearables that appropriately inform occupants of their environment and also allow them to engage with comfort parameters to certain degrees  \cite{costanza2016bit, milenkovic2013improving, kim2020designing}. Other works focus on a ``gamified" approach to engage users with their environment and activities in a socially inclusive manner \cite{mathur2015tiny, kwallek1997impact, zhong2022augmenting}. 

% Our perspective
Therefore, to our knowledge comfort studies tend to focus on designing novel interfaces and digital systems for occupants. Instead, in this position paper we wish to reconsider occupants' affective and physical comfort through the lens of materials of tangible and intangible artifacts in smart hybrid learning environment. 


\section{Material Design As a Means for Affective Comfort in Hybrid Learning Spaces}

% Why should we design for materiality
A significant consequence of smart buildings is that occupants find themselves physically immersed within an interactive object, and therefore experience interactions in a multi-sensory manner \cite{nembrini2017human}. The role of different tangible and intangible artifacts in hybrid spaces in improving affective comfort, and consequently learning-related behaviours therefore needs to be understood.  

Providing for physical and emotional comfort to learners within a smart space that is used by different people with varying usage goals remains a key question. We believe that one means of addressing this is through material design of artifacts within hybrid spaces. Moreover, materiality of hybrid spaces need to consider its role in fostering occupant interactions with non-digital elements of the building through digital means. 

% Our proposed case study
Our primary goal is to design learning spaces that can increase an occupant's subjective, affective comfort in hybrid learning spaces. As part of this objective, our first step is to identify occupants' novel experiences of comfort and emotions that arise from interactions within the built space. For this, we will conduct a case study to identify the most influential factors (both tangible and intangible) that determine the overall sense of comfort and emotions in hybrid learning spaces. The case study comprises two phases: (a) emotion and comfort label collection, and (b) building walk. 

\subsubsection{Part I: Emotion and Comfort Label collection} <--------
The goal of this phase is to obtain a large number of emotion and comfort information from occupants and users of new, and hybrid learning/working building in Europe. QR codes placed around different working, and relaxation spaces in the building will be used to ask two questions - (a) pertaining to emotions (b) pertaining to comfort. The emotion question would ask “How does this space make you feel?”. Next, users will be asked to reflect upon their comfort in the space “Are you comfortable in this space? Consider things such as temperature, light, noises etc.”. Both questions have an open-ended response format.

\subsubsection{Phase II: Building Walk}
The goal of the building walk is to have a more in-depth understanding of occupants assessment - mental, emotional and physical towards the spaces in LAB42. Phase one focuses on vocabulary while Phase two focuses on context surrounding the perception of emotions and comfort. The building walk comprises three main sessions - (a) a pre-walk briefing session, (b) walking session, and (c) post-walk session. Participants in small groups will be lead by researchers along a pre-planned route of the building (route planned by the participants). During the route, researchers engage participants in conversations about each space marked on the blueprints. Conversation  prompts include asking about impressions of the space (what is your first impression of this space? ), usage of space (How would you use this space? Why do you use this space? What kind of tasks would you perform in this space?), emotions in the space (How do you feel in this space emotionally? OR What emotions do you associate with this space? ), comfort (Is this space comfortable?	Is there sufficient light, warmth, ventilation?). Participants will also be encoraged to take polaroid photos of elements and artifacts that stabnd out to them. 

\subsubsection{Analysis}
Emotion and comfort data collected from Phase I will be analysed using Desmet et al., lexcial analysis methodology to identify and extract comfort and emotion labels. The text from building walk will be also analysed using thematic analysis and used to support the emotion labels collected from the phase 1. Additional context to emotion and comfort labels will also be obtained from the detailed text descriptions.

% Expected outcomes that can inform B X M research - Material as a catalyst for human action
\subsection{Expected Contributions}
Through our case study (comfort and emotion data collection and building walk), we aim to (among other goals) identify specific artifacts both tangible and intangible, and the properties of these artifacts that impact building occupants' comfort. Additionally, we expect to gain an understanding of the lived-in bodily comfort and emotional experiences within smart learning spaces. We believe that our findings from the proposed case study will aid the body of work within HCI by first and foremost bringing together research communities of material and interaction design with comfort studies and smart buildings and consider materiality of built environments through the lens of comfort and affect. 

We believe that spaces (and the artifacts they encompass) need to be tailored to hybrid learning spaces. Our findings may be used for the design and development of materials that can help create experiences that occupants might find lacking in smart hybrid learning spaces (for eg., sense of autonomy and control in an automated building), or enhance certain other experiences (for eg., feeling of groundedness and familiarity in a space used by many). 


% We find the growing need to foster collaboration between design,   findings may be used to initiate discussions between designers, architects and systems towards  designing materials that elicit mental and physical experiences of comfort, social engagement, privacy, motivation, groundedness, sentiments and moods typically associated with hybrid learning spaces.  



% Examples of how materiality design can aid in comfort studies
Materials that in-situ infer the environmental qualities and change attributes to provide comfort to participants. Materials that in-situ infer occupants affective states, and allow occupants to become aware of such states in a hybrid environment. 

The nature of hybriod learning spaces involves mixed activities that often occur in the same space at a time. Understanding the nature of a space to judge its suitablility of use such as for learning or studying is an important concern. Materials need to be designed that convey such information to occupants or 

participants emotions and provide subjective comfort to participants as occupants' emotions play a role in their perception of comfort and consequently their awareness and behaviour within a space.,



% materials that enable occupants to visualise comfort parameters of an area, materials that enable privacy indication, of experience of privacy in a mixed learning environment., materials that enable and encourage learners to communicate and socialise. Materials that identify qualities of a space such as learnability of a mixed use table. 


\section{Conclusion}
The increased complexity of built environments in recent years can be attributed to a growing expectation for buildings to adapt to changing socio-environmental parameters - human needs, climate, and economy. Given the inevitable growth of smart built environments, in this position paper we suggest that materials of a space can play a central role in understanding and shaping occupants comfort and emotions towards more human-centric smart buildings. We discuss why materiality (both tangible and intangible) needs to be examined and designed for occupants subjective affective and physical comfort in hybrid learning spaces. To that end, we also describe a case study that will help identify artifacts (tangible and intangible) and their specific properties that impact building occupants and therefore needs rethinking, redesign such that they may serve as catalysts for human comfort and mental health and well being. 

qualitative first step

\bibliographystyle{ACM-Reference-Format}
\bibliography{affective-comfort}

%%
%% If your work has an appendix, this is the place to put it.

\end{document}
\endinput
%%
%% End of file `sample-authordraft.tex'.


% \section{Figures}

% The ``\verb|figure|'' environment should be used for figures. One or
% more images can be placed within a figure. If your figure contains
% third-party material, you must clearly identify it as such, as shown
% in the example below.
% \begin{figure}[h]
%   \centering
%   \includegraphics[width=\linewidth]{sample-franklin}
%   \caption{1907 Franklin Model D roadster. Photograph by Harris \&
%     Ewing, Inc. [Public domain], via Wikimedia
%     Commons. (\url{https://goo.gl/VLCRBB}).}
%   \Description{A woman and a girl in white dresses sit in an open car.}
% \end{figure}

% \subsection{The ``Teaser Figure''}

% A ``teaser figure'' is an image, or set of images in one figure, that
% are placed after all author and affiliation information, and before
% the body of the article, spanning the page. If you wish to have such a
% figure in your article, place the command immediately before the
% \verb|\maketitle| command:
% \begin{verbatim}
%   \begin{teaserfigure}
%     \includegraphics[width=\textwidth]{sampleteaser}
%     \caption{figure caption}
%     \Description{figure description}
%   \end{teaserfigure}
% \end{verbatim}
