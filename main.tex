%% The first command in your LaTeX source must be the \documentclass command.
%%%% Small single column format, used for CIE, CSUR, DTRAP, JACM, JDIQ, JEA, JERIC, JETC, PACMCGIT, TAAS, TACCESS, TACO, TALG, TALLIP (formerly TALIP), TCPS, TDSCI, TEAC, TECS, TELO, THRI, TIIS, TIOT, TISSEC, TIST, TKDD, TMIS, TOCE, TOCHI, TOCL, TOCS, TOCT, TODAES, TODS, TOIS, TOIT, TOMACS, TOMM (formerly TOMCCAP), TOMPECS, TOMS, TOPC, TOPLAS, TOPS, TOS, TOSEM, TOSN, TQC, TRETS, TSAS, TSC, TSLP, TWEB.
% \documentclass[acmsmall]{acmart}

%%%% Large single column format, used for IMWUT, JOCCH, PACMPL, POMACS, TAP, PACMHCI
% \documentclass[acmlarge,screen]{acmart}

%%%% Large double column format, used for TOG
% \documentclass[acmtog, authorversion]{acmart}

%%%% Generic manuscript mode, required for submission
%%%% and peer review
\documentclass[acmconf, anonymous, review]{acmart}
\usepackage{todonotes}
\usepackage{xcolor}

%% Fonts used in the template cannot be substituted; margin 
%% adjustments are not allowed.
%%
%% \BibTeX command to typeset BibTeX logo in the docs
\AtBeginDocument{%
  \providecommand\BibTeX{{%
    \normalfont B\kern-0.5em{\scshape i\kern-0.25em b}\kern-0.8em\TeX}}}

%% Rights management information.  This information is sent to you
%% when you complete the rights form.  These commands have SAMPLE
%% values in them; it is your responsibility as an author to replace
%% the commands and values with those provided to you when you
%% complete the rights form.
\setcopyright{acmcopyright}
\copyrightyear{2018}
\acmYear{2018}
\acmDOI{XXXXXXX.XXXXXXX}

%% These commands are for a PROCEEDINGS abstract or paper.
\acmConference[Conference acronym 'XX]{Make sure to enter the correct
  conference title from your rights confirmation emai}{June 03--05,
  2018}{Woodstock, NY}


\acmBooktitle{Woodstock '18: ACM Symposium on Neural Gaze Detection,
 June 03--05, 2018, Woodstock, NY} 
\acmPrice{15.00}
\acmISBN{978-1-4503-XXXX-X/18/06}


%%
%% Submission ID.
%% Use this when submitting an article to a sponsored event. You'll
%% receive a unique submission ID from the organizers
%% of the event, and this ID should be used as the parameter to this command.
%%\acmSubmissionID{123-A56-BU3}

%%
%% end of the preamble, start of the body of the document source.
\begin{document}

%%
%% The "title" command has an optional parameter,
%% allowing the author to define a "short title" to be used in page headers.
\title{Affective Comfort in Built Hybrid Learning Spaces}

%%
%% The "author" command and its associated commands are used to define
%% the authors and their affiliations.
%% Of note is the shared affiliation of the first two authors, and the
%% "authornote" and "authornotemark" commands
%% used to denote shared contribution to the research.
\author{Shruti Rao}
\email{s.rao@uva.nl}
\orcid{1234-5678-9012}
\affiliation{
  \institution{University of Amsterdam}
  \city{Amsterdam}
  \country{The Netherlands}
}



%%
%% By default, the full list of authors will be used in the page
%% headers. Often, this list is too long, and will overlap
%% other information printed in the page headers. This command allows
%% the author to define a more concise list
%% of authors' names for this purpose.
\renewcommand{\shortauthors}{Rao et al.}

%%
%% The abstract is a short summary of the work to be presented in the
%% article.
\begin{abstract}
%   Abstracts should be about 150 words.
Some Abstract
\end{abstract}

%%
%% The code below is generated by the tool at http://dl.acm.org/ccs.cfm.
%% Please copy and paste the code instead of the example below.
%%

\begin{CCSXML}
<ccs2012>
 <concept>
  <concept_id>10010520.10010553.10010562</concept_id>
  <concept_desc>Computer systems organization~Embedded systems</concept_desc>
  <concept_significance>500</concept_significance>
 </concept>
 <concept>
  <concept_id>10010520.10010575.10010755</concept_id>
  <concept_desc>Computer systems organization~Redundancy</concept_desc>
  <concept_significance>300</concept_significance>
 </concept>
 <concept>
  <concept_id>10010520.10010553.10010554</concept_id>
  <concept_desc>Computer systems organization~Robotics</concept_desc>
  <concept_significance>100</concept_significance>
 </concept>
 <concept>
  <concept_id>10003033.10003083.10003095</concept_id>
  <concept_desc>Networks~Network reliability</concept_desc>
  <concept_significance>100</concept_significance>
 </concept>
</ccs2012>
\end{CCSXML}

\ccsdesc[500]{Computer systems organization~Embedded systems}
\ccsdesc[300]{Computer systems organization~Redundancy}
\ccsdesc{Computer systems organization~Robotics}
\ccsdesc[100]{Networks~Network reliability}

%%
%% Keywords. The author(s) should pick words that accurately describe
%% the work being presented. Separate the keywords with commas.
\keywords{affective computing, built environments, hybrid learning spaces, learning, materiality, body perception}

%% A "teaser" image appears between the author and affiliation
%% information and the body of the document, and typically spans the
%% page.
% \begin{teaserfigure}
%   \includegraphics[width=\textwidth]{sampleteaser}
%   \caption{Seattle Mariners at Spring Training, 2010.}
%   \Description{Enjoying the baseball game from the third-base
%   seats. Ichiro Suzuki preparing to bat.}
%   \label{fig:teaser}
% \end{teaserfigure}


%%
%% This command processes the author and affiliation and title
%% information and builds the first part of the formatted document.
\maketitle

\section{Introduction}
{\color{red}somehow state that space is a material that encompasses other touchable and non-touchable materials}
{\color{red}need to introduce concept of hybrid learning buildings.}

% Alternate introduction
Rapid digitization of buildings, particularly in the recent years have led to buildings transforming from inanimate structures into digital, interactive objects, even inciting the phrase ``when the machine becomes the building" \cite{nembrini2017human}. Such ``smart built environments" have presented an opportunity for HCI research to find solutions for humans to sustainably co-exist with buildings that encompass digital and interactive artifacts. This becomes especially important when smart buildings replace traditional learning spaces where learning, working, and leisure are all expected to co-habitate and flourish \footnote{https://lab42.uva.nl/}. We refer to these spaces as \textcolor{red}{hybrid learning spaces}. 

A significant aspect of human experience within built environments (and by extension smart environments) is derived from the physical and emotional comfort of the occupants \cite{alavi2017comfort}. Additionally, occupants emotions play a role in occupants perception of comfort and consequently their awareness and behaviour within a space. 

The consequence of smart hybrid learning spaces mean that that users get physically immersed within the interactive object, and bear the consequences of their interactions in a multi-sensory way. The impact of this on users emotions, comfort and therefore learning capabilities pose a challenge.  


In this position paper, we consider the materiality of a hybrid learning space that encapsulates both tangible artifacts (for eg., hardware, construction elements, furniture, etc.) and non-tangiable artifacts (for eg smell, sound, touch and temperature). We posit that materiality of smart learning environments need to be carefully examined within the context of learning and designed in a manner that positively influences occupants subjective affective states, their comfort and consequently their learning behaviors and attitudes.
                                    
We first provide a background on human building interaction (HBI) a broach realm of research. The remainder of the paper describes our case study as a first step towards understanding learning in hybrid buildings and the role materiality will play towards improved human building interactions in the context of learning spaces. 



\section{Background}
\todo[inline,color=pink]{tbd}
% HBI and Built Env
Human building interaction (HBI) is a burgeoning area of research that focuses on capturing, understanding and enhancing human interactions and experiences both with and within ``smart built environments" \cite{alavi2016future}. The primary goal of HCI being a framework that can be used to understand, compare, and converge research efforts from the fields of Human Computer Interaction (HCI), design, and Architecture in envisioning and shaping the future of living spaces and all that they encompass \cite{nembrini2017human, alavi2018artifacts}. 


% Comfort and Affect in Built Environment
A significant aspect of human experience within built environments is derived from the physical and mental comfort of the occupants \cite{alavi2017comfort}. Occupants' comfort as a result of the built environment focuses on the physiological responses of occupants to environmental factors of thermal, visual, air quality and sound. Moreover, occupants emotions play a role in occupants perception of comfort and consequently their awareness and behaviour within a space. Given that people spend a significant amount of time within built spaces, designing spaces considering the impact that it may have on occupants’ comfort is an important challenge for HBI. 

The design of engaging interaction scenarios varies at least in four directions: a) "Ludic Design"–creating a medium for occupants to become curious, exploratory, and refective with their environments and become thoughtful on how to
make changes [16], b) "Optimally Informative"–making aware IEQ
measurements through the most comprehensible and accessible
means [10, 20, 21, 26, 30, 48, 52], c) "Gamifed"–linking with social
inclusiveness to strike a better connection between occupants’ comfort and ofce activities [28], d) "General awareness"–pursuing a
long-term impact on users via educational or peripheral communication [22, 44].

Augmenting human sensors
1. Here maybe talk about Sailins work and other wearables?


\section{Designing for Affective Comfort in Hybrid Learning Spaces}

\todo[inline, color=yellow]{Why does physical and mental comfort matter in built environments ie the hybrid learning space}

\todo[inline, color=yellow]{Proposed case study}
We wish to design learning spaces that can increase an occupant's subjective, affective comfort in hybrid learning spaces. As part of this objective our first step is to identify occupants experiences of comfort and emotions that arise from interactions within the built space - that is to identify the most influential factors (both tangialble and in-tangiable) that determine the overall sense of comfort in hybrid learning spaces. For this we are  conducting a case study to establish the discourse of the language of emotion and comfort in built learning environments. The case study comprises two phases: (a) broad emotion and comfort label collection, and (b) building walk. 

\subsubsection{Phase I: Emotion and Comfort Label collection}
The goal of this phase twofold: (a) obtain a large number of emotion labels from occupants and users of Lab42, and (b) recruit neurotypicals and neurodiverse who are familiar with Lab42 for the involved building walk study. QR codes will be placed around diverse working, and relaxation spaces in the building. Upon scanning the code, users will have to answer two main questions - (a) pertaining to emotions (b) pertaining to comfort. The emotion question would ask “How does this space make you feel?”. Next, users will be asked to reflect upon their comfort in the space “Are you comfortable in this space? Consider things such as temperature, light, noises etc.”. Both questions have an open-ended response format.

\subsubsection{Phase II: Building Walk}
The goal of the building walk is to have a more in-depth understanding of occupants assessment - mental, emotional and physical towards the spaces in LAB42. Phase one focuses on vocabulary while Phase two focuses on context surrounding the perception of emotions and comfort. The building walk comprises three main sessions - (a) a pre-walk briefing session, (b) walking session, and (c) post-walk session. Participants in small groups will be lead by researchers along a pre-planned route of the building (route planned by the participants). During the route, researchers engage participants in conversations about each space marked on the blueprints. Conversation  prompts include asking about impressions of the space (what is your first impression of this space? ), usage of space (How would you use this space? Why do you use this space? What kind of tasks would you perform in this space?), emotions in the space (How do you feel in this space emotionally? OR What emotions do you associate with this space? ), comfort (Is this space comfortable?	Is there sufficient light, warmth, ventilation?). Participants will also be encoraged to take polaroid photos of elements and artifacts that stabnd out to them. 

\subsubsection{Analysis}
Emotion and comfort data collected from Phase I will be analysed using Desmet et al., lexcial analysis methodology to identify and extract comfort and emotion labels. The text from building walk will be also analysed using thematic analysis and used to support the emotion labels collected from the phase 1. Additional context to emotion and comfort labels will also be obtained from the detailed text descriptions.

\todo[inline, color=yellow]{Expected outcomes from study}
\subsubsection{Expected Contributions}
Identification of artifacts that can impact comfort and emotions in a hybrid learning space. 

\todo[inline,color=yellow]{How the outcomes/findings can inform BxM}

Signals from different sensory modalities, such as sounds [54], haptic [23], visual [23], smell [4], taste
and texture stimuli [37] can be used to transform embodied experiences, which in turn may impact on motor, social and
emotional functioning. We aim to address how materiality can transform (i.e., enhance, empower, amplify or augment) people’s perceptions of bodies given the built space and understand the relation between AI, materiality and people’s comfort in hybrid learning environments. 

The interactive experience and perception of physical and emotional comfort can be obtained through materiality tailored to such spaces that elicit mental and physical experiences such as comfort, social engagement, privacy, motivation etc - sentiments moods all associated with hybrid learning spaces.  Through our building walk and emotion and comfort label collection, we aim to (among other goals) identify specific, artifacts both tangible and intangible and the specific properties of the artifacts that impact building occupants and therefore needs rethinking, redesign such that they may serve as catalysts for human comfort and mental health and well being. 





\section{Conclusion}
\todo[inline,color=green!20!white]{Conclusion}
Given the inevitable growth of smarter built environments, we postulate and propose solutions to address potential issues, such that these built environments may be designed keeping in mind the subjective emotional and physical comfort of occupants and learners towards more human-centric, buildings. 

\todototoc
\listoftodos{list of todos}



\bibliographystyle{ACM-Reference-Format}
\bibliography{affective-comfort}

%%
%% If your work has an appendix, this is the place to put it.

\end{document}
\endinput
%%
%% End of file `sample-authordraft.tex'.


% \section{Figures}

% The ``\verb|figure|'' environment should be used for figures. One or
% more images can be placed within a figure. If your figure contains
% third-party material, you must clearly identify it as such, as shown
% in the example below.
% \begin{figure}[h]
%   \centering
%   \includegraphics[width=\linewidth]{sample-franklin}
%   \caption{1907 Franklin Model D roadster. Photograph by Harris \&
%     Ewing, Inc. [Public domain], via Wikimedia
%     Commons. (\url{https://goo.gl/VLCRBB}).}
%   \Description{A woman and a girl in white dresses sit in an open car.}
% \end{figure}

% \subsection{The ``Teaser Figure''}

% A ``teaser figure'' is an image, or set of images in one figure, that
% are placed after all author and affiliation information, and before
% the body of the article, spanning the page. If you wish to have such a
% figure in your article, place the command immediately before the
% \verb|\maketitle| command:
% \begin{verbatim}
%   \begin{teaserfigure}
%     \includegraphics[width=\textwidth]{sampleteaser}
%     \caption{figure caption}
%     \Description{figure description}
%   \end{teaserfigure}
% \end{verbatim}
